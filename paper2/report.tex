\documentclass[sigconf]{acmart}

\input{format/i523}

\begin{document}
\title{Big Data Analysis for Computer Network Defense}

\author{Jordan Simmons}
\affiliation{%
  \institution{Indiana University}
  \streetaddress{Smith Research Center}
  \city{Bloomington} 
  \state{IN} 
  \postcode{47408}
  \country{USA}}
\email{jomsimm@iu.edu}


% The default list of authors is too long for headers}
\renewcommand{\shortauthors}{G. v. Laszewski}


\begin{abstract}
Computer security threats and attacks are constantly evolving. Everyday, hackers are creating new techniques to bypass
network security for the purpose of malicious attacks. To keep up with the changing intrusion technologies, the technologies that defend these attacks need to constantly evolve also. Modern day technologies use deep learning techniques to monitor network activity, and detect malicious code. We will provide an overview of network security and modern technologies being used to protect computer systems and networks.
\end{abstract}

\keywords{i523,HID336, Computer Network Security, Big Data Analysis, Deep Learning, Intrusion Detection Systems, }


\maketitle

\section{Introduction}

Everyday a different computer network is being breached with the intent to cause harm to the system or to steal valuable data. Computer hackers are constantly creating new ways to evade network security and create malicious code that can not be detected by security systems. As malicious technologies continue to advance, the technologies that defend against these technologies need to adapt with these advances. The problem with computer network defence is that the technologies used to breach systems constantly change. Once a solution is created to defend a technology, a new malicious technology could be created the next day. Today many security specialist are using deep learning technologies to monitor network intrusions, and detect malicious code. In order to better understand computer network defense, an overview of modern attacks, network data collection processes, and the technologies used to analyze network data is provided. 


\section{Modern Network Attacks}



\section{Network Data Collection}

In case you need to create tables, you can do this with online tools
(if you do not mind sharing your data) such as
\url{https://www.tablesgenerator.com/} or other such tools (please
google for them). They even allow you to manage tables as CSV.

or generate them by hand while using the provided template in Table\ref{t:mytable}. Not ethat
the caption is before the tabular environment.



\section{Network Data Analysis}



\section{Conclusion}




\begin{acks}

  The authors would like to thank Dr. Gregor von Laszewski for his
  support and suggestions to write this paper.

\end{acks}

\bibliographystyle{ACM-Reference-Format}
\bibliography{report} 

\appendix

We include an appendix with common issues that we see when students
submit papers. One particular important issue is not to use the
underscore in bibtex labels. Sharelatex allows this, but the
proceedings script we have does not allow this.

When you submit the paper you need to address each of the items in the
issues.tex file and verify that you have done them. Please do this
only at the end once you have finished writing the paper. To d this
cange TODO with DONE. However if you check something on with DONE, but
we find you actually have not executed it correcty, you will receive
point deductions. Thus it is important to do this correctly and not
just 5 minutes before the deadline. It is better to do a late
submission than doing the check in haste. 

\section{Issues}

\TODO{Have you written the report in the specified format? - missing references section, in-text citations are [?]}



\end{document}
